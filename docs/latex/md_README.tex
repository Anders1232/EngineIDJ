Repositório do jogo final de I\+D\+J 2017-\/1. Precisamos decidir a licensa

Acompanhamento do projeto\+: \href{https://github.com/Anders1232/EngineIDJ/projects/1}{\tt https\+://github.\+com/\+Anders1232/\+Engine\+I\+D\+J/projects/1}

Sistema de uso do repositório(em adaptação)\+:
\begin{DoxyItemize}
\item Branch dev saindo de master.
\item Um branch para cada feature saindo de dev. O branch será nomeado \char`\"{}feature/$\ast$\char`\"{} onde \char`\"{}$\ast$\char`\"{} será o nome da feature. Exemplo \char`\"{}feature/drag-\/n-\/drop\char`\"{}.
\item Para juntar qualquer feature ao dev, será um pull request que deve ser revisado e testado por pelo menos um outro programador, ou seja, você nunca aceitará seus próprios pull requests. O branch será excluído assim que o pull request for aceito e realizado.
\item Se uma feature for complexa (por exemplo U\+I), vc pode fazer sub branches. Exemplo \char`\"{}feature/\+U\+I/animacao-\/telas\char`\"{}. Todas as regras acima se aplicam.
\item Semanalmente, ou quinzenalmente, faríamos um merge do tweaks (descrito mais abaixo) pro master e depois do dev pro master, atualizando o número da versão minor. E depois um merge do master pro dev.
\item A documentação deve estar atualizada e refletindo com precisão o branch dev A\+N\+T\+E\+S do merge ser feito.
\item Extremamente recomendável de que cada pull request de features já estejam com sua documentação correta.
\item Merges e pull requests só devem ser feitos se o código compilar e executar sem game-\/breaking bugs.
\item Branches de correção de bugs saem de master com nomenclatura \char`\"{}bugfix/$\ast$\char`\"{}. Onde $\ast$ é o número da issue no github. Exemplo \char`\"{}bugfix/32\char`\"{}.
\item Branches de bugfix se juntam ao master da mesma forma que um de feature se junta ao dev. Mas atualizando a versão de patch.
\item Branches de adição ou grande alterações de arte/música serão em branches de feature, preferencialmente compostos de um único commit. Numeração de versão segue o mesmo padrão.
\item Tweaks será um branch saindo de master. Ele será feito de commits únicos que ajustam coisas. Tipo nome de arquivo de música, texto que aparece sei lá onde, etc.
\item O branch tweaks não receberá um merge do master toda vez, somente pela primeira vez que for ter um patch naquela versão. Cada commit altera o número da versão de patch.
\item A numeração de versão será semântico, ou seja, M\+A\+J\+O\+R.\+M\+I\+N\+O\+R.\+P\+A\+T\+C\+H.\+T\+W\+E\+A\+K . Qualquer atualização em um número zera todos os números à sua direita.
\item Commits D\+E\+V\+E\+M ter nomes (e, se aplicáveis, resumos) descritivos. Nada de \char`\"{}asdfgh\char`\"{} ou \char`\"{}teste\char`\"{}, etc...
\item Se possível, colocar um resumo de tudo que foi alterado num pull request.
\item Entregáveis receberão etiquetas (tags) com o valor 30, 70 ou 100. 
\end{DoxyItemize}